%% LaTeX2e class for student theses
%% sections/abstract_en.tex
%% 
%% Karlsruhe Institute of Technology
%% Institute for Program Structures and Data Organization
%% Chair for Software Design and Quality (SDQ)
%%
%% Dr.-Ing. Erik Burger
%% burger@kit.edu
%%
%% Version 1.3.5, 2020-06-26

\Abstract

Consumers in the domain of healthcare and finance are generally unable to trace the provenance of their personal data and are often unaware of how their information is shared, accessed or processed by third parties, government or international organisations. At the same time, distributed ledger technologies are on the rise and steadily becoming more versatile in terms of applicable use cases. They encapsulate unique characteristics including decentralisation, immutability, transparency, trustworthiness and provenance, which can help create a network of traceability and openness. This will allow patients and customers to own their data and be in control of how their information is used, by whom and for what purpose, as well as improve their trust, reduce fraud, corruption and remove the role of central authority over the global healthcare and financial data.

In this work we investigate the suitability of two distributed ledger technology designs (Hyperledger Fabric and Ethereum) for personal data provenance in the domain of healthcare and finance. Through a detailed analysis of the available literature on distributed ledgers, data provenance, healthcare and finance, we were able to determine important personal data provenance requirements and also extract essential distributed ledger technology characteristics, considered beneficial for the use cases in this work. Through analysing, classifying and mapping requirements and characteristics between the different use cases and approaches, we aimed to provide an overview of the emerged relationships, the requirements' potential satisfiability, significance and, most importantly, the distributed ledger technology characteristics' suitability for personal data provenance.