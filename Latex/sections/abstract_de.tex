%% LaTeX2e class for student theses
%% sections/abstract_de.tex
%% 
%% Karlsruhe Institute of Technology
%% Institute for Program Structures and Data Organization
%% Chair for Software Design and Quality (SDQ)
%%
%% Dr.-Ing. Erik Burger
%% burger@kit.edu
%%
%% Version 1.3.5, 2020-06-26

\Abstract
Nutzer im Bereich Gesundheitswesen und Finanzen sind im Allgemeinen nicht in der Lage, die Provenance ihrer persönlichen Daten nachzuvollziehen, und wissen oft nicht, wie ihre Informationen von Dritten, Regierungen oder internationalen Organisationen geteilt, abgerufen oder verarbeitet werden. Gleichzeitig sind Distributed-Ledger-Technologien auf dem Vormarsch und werden in Bezug auf anwendbare Anwendungsfälle immer vielseitiger. Sie beinhalten einzigartige Merkmale wie Dezentralisierung, Unveränderlichkeit, Transparenz, Vertrauenswürdigkeit und Provenance, die dazu beitragen können, ein rückverfolgbares und offenes Netzwerk zu schaffen. Dies wird es Patienten und Kunden ermöglichen, ihre Daten zu besitzen und die Kontrolle darüber zu haben, wie ihre Informationen von wem und zu welchem Zweck verwendet werden, sowie ihr Vertrauen zu stärken, Betrug und Korruption zu reduzieren und die Rolle der zentralen Autorität über das globale Gesundheitswesen und Finanzdaten zu beseitigen.

In dieser Arbeit untersuchen wir die Eignung von zwei Arten von Distributed-Ledger-Technologien (Hyperledger Fabric und Ethereum) für die Provenance von persönlichen Daten im Bereich Gesundheitswesen und Finanzen. Durch eine detaillierte Analyse der verfügbaren Literatur zu Distributed Ledgers, Data Provenance, Gesundheitswesen und Finanzen konnten wir wichtige Anforderungen an die Provenance von persönlichen Daten ermitteln und auch wesentliche Merkmale der Distributed-Ledger-Technologien extrahieren, die für die Anwendungsfälle in dieser Arbeit als vorteilhaft erachtet werden. Durch die Analyse, Klassifizierung und Zuordnung von Anforderungen und Merkmalen zwischen den verschiedenen Anwendungsfällen und Ansätzen wollten wir einen Überblick über die entstandenen Beziehungen, die potenzielle Erfüllbarkeit, Bedeutung und, vor allem, die Eignung der Merkmale der Distributed-Ledger-Technologien für Provenance von persönlichen Daten geben.

